%%% Hlavní soubor. Zde se definují základní parametry a odkazuje se na ostatní části. %%%

%% Verze pro jednostranný tisk:
% Okraje: levý 40mm, pravý 25mm, horní a dolní 25mm
% (ale pozor, LaTeX si sám přidává 1in)
\documentclass[12pt,a4paper]{report}
\setlength\textwidth{145mm}
\setlength\textheight{247mm}
\setlength\oddsidemargin{15mm}
\setlength\evensidemargin{15mm}
\setlength\topmargin{0mm}
\setlength\headsep{0mm}
\setlength\headheight{0mm}
% Doporučená úprava zmiňuje řádkování 1.5, to ale není pro TeX relevantní.
% \openright zařídí, aby následující text začínal na pravé straně knihy
\let\openright=\clearpage

%% Pokud tiskneme oboustranně:
% \documentclass[12pt,a4paper,twoside,openright]{report}
% \setlength\textwidth{145mm}
% \setlength\textheight{247mm}
% \setlength\oddsidemargin{14.2mm}
% \setlength\evensidemargin{0mm}
% \setlength\topmargin{0mm}
% \setlength\headsep{0mm}
% \setlength\headheight{0mm}
% \let\openright=\cleardoublepage

% Přepneme na českou sazbu
\usepackage[czech]{babel}
\usepackage[IL2]{fontenc}

%% Použité kódování znaků: obvykle latin2, cp1250 nebo utf8:
\usepackage[utf8]{inputenc}

%% Ostatní balíčky
\usepackage{graphicx}
\usepackage{amsthm}

%%% Údaje o práci

% Název práce v jazyce práce (přesně podle zadání)
\def\NazevPrace{Název práce}

% Název práce v angličtině
\def\NazevPraceEN{Name of thesis}

% Jméno autora
\def\AutorPrace{Jméno Příjmení}

% Rok odevzdání
\def\RokOdevzdani{ROK}

% Název katedry nebo ústavu, kde byla práce oficiálně zadána
% (dle Organizační struktury MFF UK, případně plný název pracoviště mimo MFF)
\def\Katedra{Název katedry nebo ústavu}
\def\KatedraEN{Name of the department}

% Jedná se o katedru (department) nebo o ústav (institute)?
\def\TypPracoviste{Katedra}
\def\TypPracovisteEN{Department}

% Vedoucí práce: Jméno a příjmení s~tituly
\def\Vedouci{Vedoucí práce}

% Pracoviště vedoucího (opět dle Organizační struktury MFF)
\def\KatedraVedouciho{katedra}
\def\KatedraVedoucihoEN{department}

% Studijní program a obor
\def\StudijniProgram{studijní program}
\def\StudijniObor{studijní obor}

% Nepovinné poděkování (vedoucímu práce, konzultantovi, tomu, kdo
% zapůjčil software, literaturu apod.)
\def\Podekovani{%
Poděkování.
}

% Abstrakt (v rozsahu cca 80-200 slov; nejedná se o zadání práce)
\def\Abstrakt{%
Abstrakt.
}
\def\AbstraktEN{%
Abstract.
}

% 3 až 5 klíčových slov, každé uzavřeno ve složených závorkách
\def\KlicovaSlova{%
{klíčová} {slova}
}
\def\KlicovaSlovaEN{%
{key} {words}
}

%% Balíček hyperref, kterým jdou vyrábět klikací odkazy v PDF,
%% ale hlavně ho používáme k uložení metadat do PDF (včetně obsahu).
\usepackage[pdftex,unicode]{hyperref}   % Musí být za všemi ostatními balíčky
\hypersetup{breaklinks=true}
\hypersetup{pdftitle={\NazevPrace}}
\hypersetup{pdfauthor={\AutorPrace}}
\hypersetup{pdfkeywords=\KlicovaSlova}

%% Titulní strana a různé povinné informační strany
\input titulka.tex

%%% Strana s automaticky generovaným obsahem bakalářské práce. U matematických
%%% prací je přípustné, aby seznam tabulek a zkratek, existují-li, byl umístěn
%%% na začátku práce, místo na jejím konci.

\openright
\pagestyle{plain}
\setcounter{page}{1}
\tableofcontents

%%% Jednotlivé kapitoly práce jsou pro přehlednost uloženy v samostatných souborech
\include{uvod}
\include{kap1}
\include{kap2}

% Ukázka použití některých konstrukcí LateXu (odkomentujte, chcete-li)
% \include{ukazka}

\include{zaver}

%%% Seznam použité literatury
%%% Seznam pou�it� literatury je zpracov�n podle platn�ch standard�. Povinnou cita�n�
%%% normou pro bakal��skou pr�ci je ISO 690. Jm�na �asopis� lze uv�d�t zkr�cen�, ale jen
%%% v kodifikovan� podob�. V�echny pou�it� zdroje a prameny mus� b�t ��dn� citov�ny.

\def\bibname{Seznam pou�it� literatury}
\begin{thebibliography}{99}
\addcontentsline{toc}{chapter}{\bibname}

\bibitem{lamport94}
  Leslie Lamport,
  \emph{\LaTeX: A Document Preparation System}.
  Addison Wesley, Massachusetts,
  2nd Edition,
  1994.

\end{thebibliography}


%%% Obrázky v bakalářské práci
%%% (pokud jich je malé množství, obvykle není třeba seznam uvádět)
\listoffigures

%%% Tabulky v bakalářské práci (opět nemusí být nutné uvádět)
\listoftables

%%% Použité zkratky v bakalářské práci (opět nemusí být nutné uvádět)
\chapwithtoc{Seznam použitých zkratek}

%%% Přílohy k bakalářské práci, existují-li (různé dodatky jako výpisy programů,
%%% diagramy apod.). Každá příloha musí být alespoň jednou odkazována z vlastního
%%% textu práce. Přílohy se číslují.
\chapwithtoc{Přílohy}

\openright
\end{document}
