
% POSTER EXAMPLE
%
% This is an example of a relatively sane poster. The box structure (and the
% narrative in general) is what I would expect, but it is completely
% non-mandatory; you may include whatever you want. Preferably, erase the
% existing box structure after you read it, and start from scratch.
%
% The main communication requirements for the poster that should be satisfied
% are as such:
%
% - At the defense, it should help you talk for around 10 minutes about your
%   thesis, and convince the committee that you did something interesting and
%   sufficiently complicated. Prepare pictures that explain your main results.
%
% - It should quickly communicate the main idea of your thesis to a random
%   educated by-walker. Ideally, a moderately-witted MFF graduate who has never
%   heard about your thesis before should be able to get the main "rough idea"
%   in less than 1 minute by just looking at the poster.

% modify the fontscale parameter to make everything slighly bigger or smaller.
\documentclass[portrait,a0paper,fontscale=0.25]{baposter}

\usepackage[utf8]{inputenc}
\usepackage[T1]{fontenc}


% FONT CHOICES
% Posters do not need to be PDF/A; you can choose any relatable font from the
% TeX font catalogue without much risk. Sans-serif fonts are suggested for the
% posters; see https://tug.org/FontCatalogue/sansseriffonts.html
\usepackage[sfdefault]{Fira Sans}
%\usepackage[default]{droidsans}
%\usepackage[math]{iwona}
%\usepackage[defaultfam]{montserrat}
%\usepackage{cmbright}
%\usepackage{yfonts}\renewcommand{\familydefault}{\frakdefault}

\usepackage{color}
\usepackage{graphicx}
\usepackage{amssymb,amsmath}
\usepackage[export]{adjustbox} %allows using valign with \includegraphics
\usepackage{multicol}
\usepackage{svg}

\renewcommand{\arraystretch}{1.5}

\usetikzlibrary{positioning}

% A WORD ABOUT COLORS
%
% This template is prepared with a relatively neutral gray background that
% gives decent box borders (with white and darker gray), does not clash with
% many colors (except for violet-brown and other mushroomish colors, perhaps)
% and gives a lot of space for highlighting stuff.
%
% Generally, other color variations are good too; there are no strict rules on
% the colors. Good choices include:
%
% - white backgrounds and differentiation of box headers by color (see
%   headerFontColor)
%
% - various slightly tinted backgrounds (try red!10 instead of black!3)
% 
% - dark backgrounds
%
% Keep in mind:
% - The normal "informative" text and figures should be DARK on LIGHT
%   background, not the other way around.
%
% - If you want a dark background, soften (darken) the box backgrounds a bit so
%   that they do not "shine" too much from the poster. Use \color{white} for
%   the heading, and switch the UK/MFF logos to white (see contents of logos/).
%
% - Do not mix too many color hues together. Most hues have their widely
%   accepted meaning (green: good result, red: problem, blue: information,
%   yellow: highlighter, brown: serious problem, violet: something really
%   weird/interesting/magic, depending on the shade).

\begin{document}

\color{black!80} % default font color
\begin{poster}{grid=false,
	eyecatcher=true,
	background=plain,
	bgColorOne=black!3, % background color
	columns=2,
	headerborder=none,
	textborder=none,
	headershape=rectangle,
	headershade=plain,
	boxshade=plain,
	boxColorOne=white,
	headershade=plain,
	headerColorOne=black!15, % box header background color
	headerFontColor=black,
	}%
	{\includegraphics[height=7em]{logos/mff-black.pdf}}
	{\LARGE Streamlining Usability of Enterprise
	Data Quality Management Tools for
	Data Engineers}
	{\vspace{1ex} Zdeněk Tomis}
	{\includegraphics[height=7em]{logos/uk-red.pdf}}


%
% LEFT COLUMN
%

\begin{posterbox}[name=background,column=0]{Background}
	\textbf{Data Quality Management (DQM)} ensures data accuracy, completeness, and reliability for decision-making, regulatory compliance, and operational efficiency. DQM processes are crucial for leveraging data effectively in organizations. \\

    \vspace{0.5em}
    \textbf{Data Quality Rules} validate and cleanse data by checking formats, ensuring uniqueness, consistency, completeness, and accuracy. Integrating these rules into workflows maintains high data quality and reliability.
\end{posterbox}

\begin{posterbox}[column=0, name=ata, below=background]{Ataccama Data Quality Rules}

	\paragraph{Ataccama ONE} uses data quality rules to enforce standards on data. These rules can be integrated into automated workflows for continuous monitoring and enforcement. \\
	\paragraph{Example Rule Expression:}
	\begin{verbatim}
	NOT (lower(continent) IN {"asia", "africa", 
		"europe", "north america", "south america", 
								"oceania", "antarctica"})
	\end{verbatim}
	This expression checks if the value of the "continent" field is valid by comparing it against a predefined list of continents.
	
	\end{posterbox}
	

\begin{posterbox}[column=0, name=goals, below=ata, headerColorOne=cyan!60, boxColorOne=cyan!20]{Thesis goals}
 \textbf{Programmatic Access}: Reimplement Ataccama Expression Language in Python. Enable data engineers to execute Ataccama’s data quality rules directly within Python. \\

 \textbf{Compatibility with Ataccama}: Ensure compatibility with Ataccama’s data quality rules and expressions. \\

 \textbf{Performance Evaluation}: Test the viability of this implementation through performance comparisons with similar solutions, namely Great Expectations and Soda Core. Viable solution should be at most 10x-20x slower. \\

\end{posterbox}


\begin{posterbox}[column=0, name=summary, below=goals]{Summary}
	I developed a transpiler for a grammar to Python code using ANTLR. This tool facilitates the conversion of Ataccama Expression Language into executable Python code.

	\vspace{0.5em}

    I implemented around 100 functions in Python to replicate the functionality of Ataccama's data quality rules. These implementation use the standard Python library along with various other dependecies.

	\vspace{0.5em}

    I created a comprehensive set of test suites containing over 1000 test cases in more than 200 test methods. These tests ensure compatibility with Ataccama's runtime.

\end{posterbox}


%
% FOOTER
%

% \begin{posterbox}[column=0, span=2, name=footer, below=something2,
% 	textborder=none, headerborder=none, boxheaderheight=0pt,
% 	boxColorOne=black!3]{}
% If some institute/grant/department sponsored the work, put an acknowledgement here.
% \end{posterbox}

%
% RIGHT COLUMN
%
% It is usually best to fill most of the poster with your results and
% conclusions. Again, use simple annotated pictures wherever possible. Plots
% with measurements are perfect, tables are also good.
%

\begin{posterbox}[column=1, name=result1]{Performance analysis results}


\begin{center}
	\includesvg[width=0.68\linewidth]{plots/execution_time_comparison_continents.svg}
\end{center}


\begin{center}
\includesvg[width=0.68\linewidth]{plots/relative_speedup_comparison_continents.svg}
\end{center}


\begin{center}
	\includesvg[width=0.68\linewidth]{plots/execution_time_comparison_customers.svg}
\end{center}


\begin{center}
\includesvg[width=0.68\linewidth]{plots/relative_speedup_comparison_customers.svg}
\end{center}

\end{posterbox}


\begin{posterbox}[column=1, name=conclusion, below=result1, bottomaligned=summary]{Performance analysis conclusion}
	The Python implementation's performance was within the target of being  \textbf{no more than 10-20 times slower} than existing solutions, meeting the viability criteria.
\end{posterbox}



\begin{posterbox}[column=0, name=footer, below=summary,span=2]{Information}
	\begin{multicols}{2}
		\raggedright
		Contact: Zdeněk Tomis, \texttt{https://zdenektomis.eu} \\
        Supervisor: doc. Ing. Lubomír Bulej, Ph.D. \\
        Department of Distributed and Dependable Systems \\
		\columnbreak
		\centering


			\raisebox{10em}{
		
				\includesvg[height=2em]{img/logo-dark.svg} 
			}
		

	\end{multicols}

\end{posterbox}

\end{poster}
\end{document}
