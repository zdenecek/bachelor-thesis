\chapter*{Introduction}
\addcontentsline{toc}{chapter}{Introduction}

\acrfull{dqm} refers to the processes, technologies, and practices used to maintain high quality in data through its lifecycle. It encompasses the acquisition, implementation, and control of data accuracy, completeness, reliability, and relevance in enterprise systems. \acrshort{dqm} ensures that data remains accurate, consistent, and accessible across all platforms and applications within an organization.

In the age of big data and advanced analytics, \acrshort{dqm} is not just a luxury—it's an imperative. It is critical for modern enterprises for many reasons, including the following:

\begin{itemize}
    \item   Informed Decision-Making
    
    High-quality data is pivotal for accuracy in decision-making. Decisions based on inaccurate or incomplete data can lead to significant financial losses and strategic missteps.

    \item   Regulatory Compliance
    
    Many industries are subject to regulations that mandate the integrity and confidentiality of data. For example, the GDPR in Europe and HIPAA in the United States impose strict guidelines on data privacy and the quality of information that is stored and processed.
        \acrshort{dqm} helps organizations comply with these regulations and avoid hefty penalties by ensuring data is managed correctly throughout its lifecycle.

    \item   Enhanced Customer Satisfaction

    Data quality directly impacts customer experience. Accurate customer data helps businesses understand their clients better, tailor their interactions more effectively, leading to improved customer satisfaction and loyalty.

    \item Operational Efficiency
    
    High-quality data reduces errors and the need for rework. For instance, accurate inventory data helps in managing stock levels efficiently, avoiding overstocking or stockouts.
    By automating data cleansing and enrichment, organizations can streamline workflows and allow employees to focus on higher-value activities rather than correcting data errors.    

    \item     Risk Mitigation

    Poor data quality is a major risk in itself—it can skew analysis, leading to misguided strategies that may harm the business.
    \acrshort{dqm} practices identify and correct discrepancies in data before they propagate through the enterprise, thereby mitigating risks associated with data handling and storage.

\end{itemize}




